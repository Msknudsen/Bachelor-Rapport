\chapter{Tilsvarende projekter \& Relevant litteratur}
\section{Tilsvarende projekter}
BIMx - Building Information Model eXplorer
En app som gør det muligt at se et kort over hvordan bygningen ser ud, både indvendigt og udvendigt.

Avenza Maps, iGIS
Begge app med en kort der indeholder koordinator man selv har sat ind, og derefter kan man få en ruteanvisning udgivet baseret på ens lokation.

Disse app's bliver nævnt som mulige appliaktioner som kan hjælpe Rambøll med at digitalisere deres bygge registreringer. Alle tre har dog problemer med at opfylde Rambølls behov, og dækker kun delvist hvad Rambøll ønsker. \newline
Disse andre produkter vil i sammenspil med Rambøll blive gennemgået for de features som Rambøll bruger i den pågældende. Disse features vil så evt. blive implenteret i Rambøll tilsyns applikation. \\

\section{Relevant litteratur}
Det litteratur der skal bruges gennem dette projekt er mest vedr. software som f.eks. litteratur omkring Xarmarin og SQL.
Det meste litteratur til disse to emner ligger frit tilgængeligt på nettet. Så det kan være links som f.eks:
\begin{itemize}
	\item Opbygning af iOS apps i Xarmarin \\
	\url{https://developer.xamarin.com/guides/ios/}
	\item Xarmarin iOS API'er \\
	\url{https://developer.xamarin.com/api/root/ios-unified/}
	\item SQL Database for mobile applikationer \\
	\url{https://docs.microsoft.com/en-us/azure/app-service-mobile/}

\end{itemize} 


