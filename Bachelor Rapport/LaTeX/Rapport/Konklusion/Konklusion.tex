\chapter{Konklusion}
Dette kapitel har til formål at beskrive den samlede 
konklusion for projektet. 

Formålet med projektet var at udvikle et system til overvågning af trafikregulering i Randers Kommune. Dette udmundede i produktet Traffic Control, som består af en ASP.NET baseret web applikation samt en Android applikation. Heri findes mulighed for at oprette brugere, ændre brugeroplysninger, oprette sager, tage sager samt ændre sager. Der er lavet et rollehierarki med tre brugertyper med forskellige rettigheder til systemet, for at forhindre uhensigtsmæssige ændringer i databasen. 

Det er lykkedes at implementere de essentielle features med høj driftsikkerhed og databeskyttelse. Ikke alle features er implementeret både på Android- og webapplikationen. 
Frontend anvender kommunikation over netværk til en velfungerende backend med serversystem bestående af web- og databaseserver.

Gruppearbejdet har båret præg af et agilt flow, men uden faste rammer. Anvendelse af Scrum blev forsøgt i starten af processen, men dette viste sig ikke egnet til gruppens arbejdsgang.

Traffic Control som prototypen er i dag et minimum viable product, og vil med en videre arbejde kunne sættes i drift inden for en overskuelig tidshorisont.
